% Compiled with lualatex.
\documentclass[a4paper,12pt,twoside]{article}

% Check for outdated packages and bad practices.
\usepackage[l2tabu, orthodox]{nag}% https://ctan.org/pkg/nag

% TODO: Fill this document with examples of trees, figures, etc.
% TODO: Create your own examples.
% TODO: Include example with metrix and lineno.
% TODO: Include examples and/or notes for all imported packages.
% TODO: Fix warnings.
% TODO: Can fancyhdr give titles based upon language?
% TODO: Display use of Norvegia.

\usepackage[verbose]{polyglossia}% https://www.ctan.org/pkg/polyglossia
% TODO: Delete after testing.
\usepackage{lipsum}

% Load lineno before csquotes.
% https://tex.stackexchange.com/a/447159
\usepackage{lineno}% https://ctan.org/pkg/lineno
\usepackage{metrix}% https://ctan.org/pkg/metrix
\usepackage{csquotes}% https://ctan.org/pkg/csquotes
\usepackage{booktabs}% https://ctan.org/pkg/booktabs

\usepackage{mybiblatex}


% Mathematical symbols.
\usepackage{amsmath}% https://ctan.org/pkg/amsmath
% Various symbols including semantics brackets.
\usepackage{stmaryrd}% https://ctan.org/pkg/stmaryrd

% https://latex-tutorial.com/tutorials/figures/
% https://latex-tutorial.com/subfigure-latex/
% https://latex-tutorial.com/figure-placement-in-text/
% https://www.overleaf.com/learn/latex/How_to_Write_a_Thesis_in_LaTeX_(Part_3)%3A_Figures%2C_Subfigures_and_Tables
\usepackage{subcaption}% https://ctan.org/pkg/subcaption

\usepackage{graphicx}% https://ctan.org/pkg/graphicx

% Draw syntactic trees.
\usepackage{myforest}

% Stylings for the header, footer, etc.
\usepackage{myfancyhdr}

% Include list of tables/figures in the TOC.
\usepackage[nottoc]{tocbibind}% https://ctan.org/pkg/tocbibind

% Various notes on gb4e.
% https://tex.stackexchange.com/questions/364163/incompatibility-between-siunitx-and-gb4e/364179#364179
% https://tex.stackexchange.com/questions/325621/gb4e-package-causing-capacity-errors/325635#325635
% https://tex.stackexchange.com/questions/121416/putting-an-underscore-in-a-label
% https://outde.xyz/2020-07-07/latex-pet-peeves.html
\makeatletter
\def\new@fontshape{}
\makeatother
\usepackage{gb4e}% https://ctan.org/pkg/gb4e
\noautomath

% Improved glosses for use with gb4e.
% Recommended by: https://amunn.github.io/assets/latex/latex-guide.pdf.
\usepackage{cgloss}% https://staticweb.hum.uu.nl/medewerkers/alexis.dimitriadis/latex/

\usepackage{outlines}% https://ctan.org/pkg/outlines
\usepackage{enumitem}% https://ctan.org/pkg/enumitem
\setlist[enumerate,1]{label=\Roman*.}
\setlist[enumerate,2]{label=\Alph*.}
\setlist[enumerate,3]{label=\roman*.}
\setlist[enumerate,4]{label=\alph*.}

\usepackage[colorlinks, citecolor=blue, linkcolor=blue]{hyperref}% https://ctan.org/pkg/hyperref

% Set PDF metadata.
\hypersetup{
  pdfinfo = {
    Title = Linguistics Paper,
    Author = {Skaffen-Amtiskaw},
    Subject = Formatting linguistics papers with LaTeX
  }
}

\makeatletter
\RenewDocumentCommand{\sectionautorefname}{}{%
  §~\@gobble
}
\makeatother
\RenewCommandCopy{\subsectionautorefname}{\sectionautorefname}
\RenewCommandCopy{\subsubsectionautorefname}{\sectionautorefname}
\def\Itemautorefname~#1\null{(#1)\null}%

% TODO: Consolidate font commands (here?).
% http://dante.ctan.org/tex-archive/info/luatex/lualatex-doc-de/lualatex-doc-DE.pdf
\defaultfontfeatures{Ligatures=TeX}
\setmainfont{FreeSerif}
\setsansfont{FreeSans}
\setmonofont{FreeMono}
% TODO: Add a Sanskrit example.
% \newfontface\sanskrit{Samyak-Devanagari.ttf}

% From the polyglossia manual, § 2.1: Note In general, it is advisable to
% activate the languages after all packages have been loaded. This is
% particularly important if you use right-to-left scripts or languages with
% babel shorthands.
\setdefaultlanguage[variant=american,ordinalmonthday=false]{english}
\setotherlanguage{japanese}
\setotherlanguage[variant=ancient]{greek}
\setotherlanguage[variant=classic]{latin}

\newfontfamily\japanesefont{TakaoMincho}
% TODO: Incorporate these commands, taken from Overleaf.
% https://www.overleaf.com/learn/latex/Japanese#luatex-ja_package_bundle_with_LuaLaTeX
% \setmainjfont{TakaoMincho}
% \setsansjfont{TakaoGothic}
% \setmonojfont{Komatuna}

% Elevate missing characters in a font from warning to error.
% The Not So Short Introduction to LaTeX, § 2.8.1.
\tracinglostchars=3

\begin{document}

\pagenumbering{roman}

\tableofcontents
\thispagestyle{toc}
\newpage

\listoffigures
\thispagestyle{lof}
\newpage

\listoftables
\thispagestyle{lot}
\newpage

\title{\thistitle}
\author{\thisauthor}
% \date{} % uncomment to add a specific date

\maketitle
\thispagestyle{plain}
\pagenumbering{arabic}

\section{Polyglossia testing}

\begin{quote}
    \begin{greek}
        μῆνιν ἄειδε θεὰ Πηληϊάδεω Ἀχιλῆος οὐλομένην, ἣ μυρί' Ἀχαιοῖς ἄλγε' ἔθηκε,
        πολλὰς δ' ἰφθίμους ψυχὰς Ἄϊδι προί̈ αψεν ἡρώων, αὐτοὺς δὲ ἑλώρια τεῦχε
        κύνεσσιν οἰωνοῖσί τε πᾶσι, Διὸς δ' ἐτελείετο βουλή, ἐξ οὗ δὴ τὰ πρῶτα
        διαστήτην ἐρίσαντε Ἀτρεί̈ δης τε ἄναξ ἀνδρῶν καὶ δῖος Ἀχιλλεύς.
    \end{greek}
\end{quote}

\begin{quote}
    \begin{latin}
        \lipsum[1]
    \end{latin}
\end{quote}

\begin{quote}
    \begin{japanese}
        暴れる
    \end{japanese}
\end{quote}

\section{Overview}

In this brief overview of things useful for linguists, we see basic biblatex
functionality in \autoref{sec:biblatex}, some symbols used in
\autoref{sec:semantics}, phonology examples in \autoref{sec:phonology},
including both Japanese in \autoref{ssec:phonology-japanese} and a table with
Old English diphthongs in \autoref{ssec:phonology-oldenglish}, a figure showing
an LFG \textit{f}-structure, and some Latin glosses.

\section{Biblatex functionality}
\label{sec:biblatex}

A bare citation command: \autocite{burgess-plunkett-2013-1}. A citation command
for use in the flow of text: As \textcite{burgess-plunkett-2013-1} said \dots~.

\section{Semantics}
\label{sec:semantics}

\autoref{ex:works-denotation} shows a little bit of semantics/mathematical
notation:\footcite[15]{heim-kratzer-1998}

\begin{exe}
    \ex\label{ex:works-denotation}

    % Make the example number be at the top of the table.
    % https://tex.stackexchange.com/a/5025
    \leavevmode\vadjust{\vspace{-\baselineskip}}

    \begin{tabular}[c]{l l}
      $\llbracket$ \textbf{works} $\rrbracket$ = f : & D $\shortrightarrow$ \{0, 1\} \\
                                                     & For all x $\in$ D, f(x) iff x works.  \\
    \end{tabular}

\end{exe}

\section{Phonology}\label{sec:phonology}
\subsection{Japanese}\label{ssec:phonology-japanese}

% https://en.wikipedia.org/wiki/Japanese_phonology
/b/ > bilabial fricative [β]: /abareru/ > [aβaɾeɾɯ] \textit{abareru} \textjapanese{暴れる} ``to behave violently''


\subsection{Japanese}\label{ssec:phonology-oldenglish}

% <Multi_key> <backslash> <at> <i> <b>	: "̯"	U032f	# COMBINING INVERTED BREVE BELOW -- ??
% <Multi_key> <colon> <plus>  	      	: "ː"	U02D0		# MODIFIER LETTER TRIANGULAR COLON
% <Multi_key> <a> <h>			: "ɑ"	U0251		# LATIN SMALL LETTER ALPHA

Old English diphthongs, as seen in \hyperref[tab:oe-diphthongs]{the table below}.

% https://latex-tutorial.com/tables-in-latex/
\begin{table}
  \caption{Old English Diphthongs}
  \label{tab:oe-diphthongs}
  \centering
  \begin{tabular}{ccc}
    \toprule
    \textbf{First element} & \textbf{Short (monomoraic)} & \textbf{Long (bimoraic)} \\
    \midrule
    \textbf{Close} & iy̯ & iːy̯ \\
    \textbf{Mid} & eo̯ & eːo̯ \\
    \textbf{Open} & æɑ̯ & æːɑ̯ \\
    \bottomrule
  \end{tabular}
\end{table}


\section{Syntax}\label{sec:syntax}

\autoref{fig:lfg} shows us an example from LFG.

\begin{figure}
  \caption{LFG example}
  \label{fig:lfg}
  \includegraphics{images/lfg.png}
\end{figure}

Now we have \hyperref[ex:mary-smiles]{a syntax tree} made with \hyperref{https://www.ctan.org/pkg/forest}{}{}{forest} and \hyperref[ex:mary-smiles-with-brackets]{the same tree with semantics brackets around it}.

\begin{minipage}{0.3\textwidth}
\begin{exe}

  \ex\label{ex:mary-smiles}
  \leavevmode\vadjust{\vspace{-\baselineskip}}

  \begin{forest}
    [S [NP [N [Mary]]] [VP [V [smiles]]]]
  \end{forest}

\end{exe}
\end{minipage}
\begin{minipage}{0.3\textwidth}
\begin{exe}

  \ex\label{ex:mary-smiles-with-brackets}
  \leavevmode\vadjust{\vspace{-\baselineskip}}

  $\left\llbracket
  \begin{forest}
    [S [NP [N [Mary]]] [VP [V [smiles]]]]
  \end{forest}
  \right\rrbracket$

\end{exe}
\end{minipage}

\section{Glosses}

Finally, some glosses.

\begin{exe}

  \ex\label{ex:lat-mir-pluper}
  \gll sē-v-erā-tis \\
  sow-PFV-PST-2PL \\
  \trans \enquote*{you all had sown}

  \ex\label{ex:lat-mir-fut-per}
  \gll sē-v-eri-tis \\
  sow-PFV-FUT-2PL \\
  \trans \enquote*{you all will have sown}

\end{exe}

% https://ctan.uib.no/macros/latex/contrib/gb4e/gb4e-doc.pdf
\begin{exe}% sets up the top-level example environment

  \ex\label{ex:here} Here is one. % example with running number
  \ex[*]{Here another is.}% judged ex. with running number
  \ex Here are some with judgements.
  \begin{xlist}% first embedding (alphabetical numbering)
    \ex[]{A grammatical sentence am I.}
    \ex[*]{An ungrammatical sentence is you.}
    \ex[??]{A dubious sentence is she.}
    \ex % just the number
    \begin{xlist}% second embedding (roman numbering)
      \ex[**]{Need one a second embedding?}
      \ex[\%]{sometime.}
    \end{xlist}% end second embedding
    \ex Dare to judge me!
  \end{xlist}% end first embedding
  \ex This concludes our demonstration.

\end{exe}% end example environment

\newpage

\printbibliography[
  heading=bibintoc,
]
\thispagestyle{bib}

\end{document}

%%% Local Variables:
%%% mode: latex
%%% TeX-master: t
%%% End:
