% Compiled with lualatex.
\documentclass[a4paper,12pt,twoside]{article}

% Check for outdated packages and bad practices.
\usepackage[l2tabu, orthodox]{nag}% https://ctan.org/pkg/nag

% TODO: Use polyglossia for Japanese, etc.
% TODO: Fill this document with examples of trees, figures, etc.
% TODO: Create your own examples.
% TODO: Include example with metrix and lineno.
% TODO: Include examples for all imported packages.
% TODO: Fix warnings.
% TODO: Can fancyhdr give titles based upon language?

\usepackage{polyglossia}% https://www.ctan.org/pkg/polyglossia

% https://9to5science.com/lineno-package-in-latex-causes-warning-message
\usepackage{metrix}% https://ctan.org/pkg/metrix
\usepackage{lineno}% https://ctan.org/pkg/lineno

\usepackage{csquotes}% https://ctan.org/pkg/csquotes
\usepackage{booktabs}% https://ctan.org/pkg/booktabs

\usepackage[style=authoryear,sorting=nyt,eprint=false,isbn=false]{biblatex}% https://ctan.org/pkg/biblatex
% ~/texmf/bibtex/bib
\bibliography{refs}

% Don't print ISBN if DOI is defined.
% https://tex.stackexchange.com/a/76542
% § 4.5.3 in the biblatex documentation.
\DeclareSourcemap{
  \maps[datatype=bibtex]{
    \map{
      \step[fieldsource=doi,final]
      \step[fieldset=isbn,null]
    }
  }
}

% TODO: Reformat, move down.
\renewcommand{\postnotedelim}{\addcolon\space}

% Strip the language field from the bibliography.
% https://tex.stackexchange.com/a/401026
\AtEveryBibitem{\clearlist{language}}
\AtEveryBibitem{\clearfield{month}}

% Make a colon (and space) the separator between *title and *subtitle.
% https://tex.stackexchange.com/a/574295
\renewcommand*{\subtitlepunct}{\addcolon\space}

% Prevent "p." from being displayed in citations with postnotes.
% https://tex.stackexchange.com/a/53055
\DeclareFieldFormat{postnote}{#1}
\DeclareFieldFormat{multipostnote}{#1}

% http://dante.ctan.org/tex-archive/info/luatex/lualatex-doc/lualatex-doc.pdf
\defaultfontfeatures{Ligatures=TeX}
\setmainfont{FreeSerif}
\setsansfont{FreeSans}
\setmonofont{FreeMono}
\newfontface\sanskrit{Samyak-Devanagari.ttf}

% https://tex.stackexchange.com/a/172742

% TODO: Should this be language specific?
% This doesn't appear to be language-specific, cf.
% IE assignment and https://tex.stackexchange.com/a/355056
% \appto\captionsnorwegian{\renewcommand\Itemautorefname{~#1\null{(#1)\null}}}
\def\sectionautorefname~#1\null{§ #1\null}%
\def\subsectionautorefname~#1\null{§ #1\null}%
\def\subsubsectionautorefname~#1\null{§ #1\null}%
\def\Itemautorefname~#1\null{(#1)\null}%

% Mathematical symbols.
\usepackage{amsmath}% https://ctan.org/pkg/amsmath
% Various symbols including semantics brackets.
\usepackage{stmaryrd}% https://ctan.org/pkg/stmaryrd

% https://latex-tutorial.com/tutorials/figures/
% https://latex-tutorial.com/subfigure-latex/
% https://latex-tutorial.com/figure-placement-in-text/
% https://www.overleaf.com/learn/latex/How_to_Write_a_Thesis_in_LaTeX_(Part_3)%3A_Figures%2C_Subfigures_and_Tables
\usepackage{subcaption}% https://ctan.org/pkg/subcaption

% TODO: Do you want borders around all figures?
% Borders around all figures.
% https://en.m.wikibooks.org/wiki/LaTeX/Floats,_Figures_and_Captions
% \usepackage{float}
% \floatstyle{boxed}
% \restylefloat{figure}

\usepackage{graphicx}% https://ctan.org/pkg/graphicx

% Draw syntactic trees.
\usepackage{myforest}

% Stylings for the header, footer, etc.
\usepackage{myfancyhdr}

% TODO: Document this.
\usepackage[nottoc]{tocbibind}% https://ctan.org/pkg/tocbibind

% Various notes on gb4e.
% https://tex.stackexchange.com/questions/364163/incompatibility-between-siunitx-and-gb4e/364179#364179
% https://tex.stackexchange.com/questions/325621/gb4e-package-causing-capacity-errors/325635#325635
% https://tex.stackexchange.com/questions/121416/putting-an-underscore-in-a-label
% https://outde.xyz/2020-07-07/latex-pet-peeves.html
\makeatletter
\def\new@fontshape{}
\makeatother
\usepackage{gb4e}% https://ctan.org/pkg/gb4e
\noautomath

% improved glosses for use with gb4e
% recommended by: https://amunn.github.io/assets/latex/latex-guide.pdf
\usepackage{cgloss}% https://staticweb.hum.uu.nl/medewerkers/alexis.dimitriadis/latex/

\usepackage{outlines}% https://ctan.org/pkg/outlines
\usepackage{enumitem}% https://ctan.org/pkg/enumitem
\setlist[enumerate,1]{label=\Roman*.}
\setlist[enumerate,2]{label=\Alph*.}
\setlist[enumerate,3]{label=\roman*.}
\setlist[enumerate,4]{label=\alph*.}

% TODO: Format command names consistently.
% A simple solution to avoid repeatedly defining the author and title in
% conjunction with fancyhdr and \maketitle.
% https://stackoverflow.com/a/6670453
\newcommand{\thistitle}{\textsc{Linguistics Paper}}
\newcommand{\thisauthor}{\textsc{Skaffen-Amtiskaw}}

\setdefaultlanguage[variant=american,ordinalmonthday=false]{english}

% Elevate missing characters in a font from warning to error.
% The Not So Short Introduction to LaTeX, § 2.8.1.
\tracinglostchars=3

% TODO: Test this placement of hyperref with gb4e.
\usepackage[colorlinks, citecolor=blue, linkcolor=blue]{hyperref}% https://ctan.org/pkg/hyperref

% Set PDF metadata.
\hypersetup{
  pdfinfo = {
    Title = Linguistics Paper,
    Author = {Skaffen-Amtiskaw},
    Subject = Formatting linguistics papers with LaTeX
  }
}

% Typeset the alphabetic part of an ordinal number in superscript.
\NewDocumentCommand{\ordabbr}{m}{%
  \textsuperscript{\underline{#1}}
}

\begin{document}

\pagenumbering{roman}

\tableofcontents
\thispagestyle{toc}
\newpage

\listoffigures
\thispagestyle{lof}
\newpage

\listoftables
\thispagestyle{lot}
\newpage

\title{\thistitle}
\author{\thisauthor}
% \date{} % uncomment to add a specific date

\maketitle
\thispagestyle{plain}
\pagenumbering{arabic}

\section{Overview}

In this brief overview of things useful for linguists, we see basic biblatex
functionality in \autoref{sec:biblatex}, some symbols used in
\autoref{sec:semantics}, phonology examples in \autoref{sec:phonology},
including both Japanese in \autoref{ssec:phonology-japanese} and a table with
Old English diphthongs in \autoref{ssec:phonology-oldenglish}, a figure showing
an LFG \textit{f}-structure, and some Latin glosses.

\section{Biblatex functionality}
\label{sec:biblatex}

A bare citation command: \autocite{burgess-plunkett-2013-1}. A citation command
for use in the flow of text: As \textcite{burgess-plunkett-2013-1} said \dots~.

\section{Semantics}
\label{sec:semantics}

\autoref{ex:works-denotation} shows a little bit of semantics/mathematical
notation:\footcite[15]{heim-kratzer-1998}

\begin{exe}
    \ex\label{ex:works-denotation}

    % Make the example number be at the top of the table.
    % https://tex.stackexchange.com/a/5025
    \leavevmode\vadjust{\vspace{-\baselineskip}}

    \begin{tabular}[c]{l l}
      $\llbracket$ \textbf{works} $\rrbracket$ = f : & D $\shortrightarrow$ \{0, 1\} \\
                                                     & For all x $\in$ D, f(x) iff x works.  \\
    \end{tabular}

\end{exe}

\section{Phonology}\label{sec:phonology}
\subsection{Japanese}\label{ssec:phonology-japanese}

% https://en.wikipedia.org/wiki/Japanese_phonology
/b/ > bilabial fricative [β]: /abareru/ > [aβaɾeɾɯ] \textit{abareru} 暴れる ``to behave violently''


\subsection{Japanese}\label{ssec:phonology-oldenglish}

% <Multi_key> <backslash> <at> <i> <b>	: "̯"	U032f	# COMBINING INVERTED BREVE BELOW -- ??
% <Multi_key> <colon> <plus>  	      	: "ː"	U02D0		# MODIFIER LETTER TRIANGULAR COLON
% <Multi_key> <a> <h>			: "ɑ"	U0251		# LATIN SMALL LETTER ALPHA

Old English diphthongs, as seen in \hyperref[tab:oe-diphthongs]{the table below}.

% https://latex-tutorial.com/tables-in-latex/
\begin{table}
  \caption{Old English Diphthongs}
  \label{tab:oe-diphthongs}
  \centering
  \begin{tabular}{ccc}
    \toprule
    \textbf{First element} & \textbf{Short (monomoraic)} & \textbf{Long (bimoraic)} \\
    \midrule
    \textbf{Close} & iy̯ & iːy̯ \\
    \textbf{Mid} & eo̯ & eːo̯ \\
    \textbf{Open} & æɑ̯ & æːɑ̯ \\
    \bottomrule
  \end{tabular}
\end{table}


\section{Syntax}\label{sec:syntax}

\autoref{fig:lfg} shows us an example from LFG.

\begin{figure}
  \caption{LFG example}
  \label{fig:lfg}
  \includegraphics{images/lfg.png}
\end{figure}

Now we have \hyperref[ex:mary-smiles]{a syntax tree} made with \hyperref{https://www.ctan.org/pkg/forest}{}{}{forest} and \hyperref[ex:mary-smiles-with-brackets]{the same tree with semantics brackets around it}.

\begin{minipage}{0.3\textwidth}
\begin{exe}

  \ex\label{ex:mary-smiles}
  \leavevmode\vadjust{\vspace{-\baselineskip}}

  \begin{forest}
    [S [NP [N [Mary]]] [VP [V [smiles]]]]
  \end{forest}

\end{exe}
\end{minipage}
\begin{minipage}{0.3\textwidth}
\begin{exe}

  \ex\label{ex:mary-smiles-with-brackets}
  \leavevmode\vadjust{\vspace{-\baselineskip}}

  $\left\llbracket
  \begin{forest}
    [S [NP [N [Mary]]] [VP [V [smiles]]]]
  \end{forest}
  \right\rrbracket$

\end{exe}
\end{minipage}

\section{Glosses}

Finally, some glosses.

\begin{exe}

  \ex\label{ex:lat-mir-pluper}
  \gll sē-v-erā-tis \\
  sow-PFV-PST-2PL \\
  \trans \enquote*{you all had sown}

  \ex\label{ex:lat-mir-fut-per}
  \gll sē-v-eri-tis \\
  sow-PFV-FUT-2PL \\
  \trans \enquote*{you all will have sown}

\end{exe}

% https://ctan.uib.no/macros/latex/contrib/gb4e/gb4e-doc.pdf
\begin{exe}% sets up the top-level example environment

  \ex\label{ex:here} Here is one. % example with running number
  \ex[*]{Here another is.}% judged ex. with running number
  \ex Here are some with judgements.
  \begin{xlist}% first embedding (alphabetical numbering)
    \ex[]{A grammatical sentence am I.}
    \ex[*]{An ungrammatical sentence is you.}
    \ex[??]{A dubious sentence is she.}
    \ex % just the number
    \begin{xlist}% second embedding (roman numbering)
      \ex[**]{Need one a second embedding?}
      \ex[\%]{sometime.}
    \end{xlist}% end second embedding
    \ex Dare to judge me!
  \end{xlist}% end first embedding
  \ex This concludes our demonstration.

\end{exe}% end example environment

\newpage

\printbibliography[
  heading=bibintoc,
]
\thispagestyle{bib}

\end{document}

%%% Local Variables:
%%% mode: latex
%%% TeX-master: t
%%% End:
